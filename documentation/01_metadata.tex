% 01_metadata.tex
% this .tex file vcovers everything from the heading down to the keyterms and footnaotes on the first page. the rest is managed at the main.tex file.

\newcommand{\Journal}{Technical Documentation}
\newcommand{\Year}{2024}
\newcommand{\Month}{May}

\newcommand{\Volume}{I}
    \newcommand{\vol}{Vol. ~\Volume}
    \newcommand{\Vol}{VOLUME~\Volume}

\newcommand{\Number}{I}
\newcommand{\TITLE}{Brass Bot : Technological Framework for Collecting Spent Cartridges}
\newcommand{\Title}{Brass Bot}
\newcommand{\TitleFoot}{\footnote[2]{Submission for \href{https://courses.dce.harvard.edu/?details&srcdb=202403&crn=34560}{DGMD $E-17$} Spring Academic Term of 2024.}}

\newcommand{\AUTHOR}{LopezGarcia, G. Ross, D. Weinberg, C.}
\newcommand{\AuthorFoot}{\footnote[1]{Grigori LopezGarcia ALM, Donald Ross ALB, Corbett Weinberg ALB}}

\newcommand{\ABSTRACT}{This project professes the challenge of designing and prototyping an autonomous device that will collect the spent cartridges that have already been ejected from a firearm and are on the floor of a firing range.

Two distinct prototypes were developed one by Grigori LopezGarcia, the seconded by Corbett Weinberg. These prototypes were designed considering the unique constraints posed by remote collaboration and geographical dispersion of a team member. Through iterative design, integration of feedback from multiple hardware tests, the project encapsulates -- working prototypes that were a focused designed on the physical hardware. Through its triumph's and failures the difficulty of the hardware process. For better, or for worse, no simulators were used in the creation of this robot.}

\newcommand{\KEYWORDS} {Three Dimensional Printing, Autonomous Robotics, Firearm-Safety, Innovative Technologies, Material Recovery, Python}