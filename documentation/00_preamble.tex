% 00_preamble.tex

\usepackage[authordate, backend=biber, natbib=true]{biblatex-chicago}
\addbibresource{bibliography.bib} % my bib file

\usepackage[T1]{fontenc}
\usepackage[utf8]{inputenc}
\usepackage[most]{tcolorbox}
\usepackage[noframe]{showframe}
\usepackage{mwe}
\usepackage{scrlayer-scrpage}
\usepackage{hyphenat}
\usepackage[english]{babel}
\usepackage{blindtext}
\usepackage{fontaxes}
\usepackage{hyperref}
\usepackage[bottom,splitrule,hang,flushmargin]{footmisc}

\usepackage[
    protrusion=true,
    expansion=true,
    spacing=true,
    tracking=true,
    kerning=true
]{microtype}

\KOMAoptions{
    paper=a5,
    fontsize=12pt,
    egregdoesnotlikesansseriftitles,
    headings=small,
    headings=optiontohead,
    BCOR=5mm,
    DIV=14,
    parskip=false,
    footnotes=multiple,
    twoside=true,
}

% HEADER
\setlength{\headsep}{1\baselineskip}
\setlength{\headheight}{3\baselineskip}
\setkomafont{pagehead}{\normalfont}
%%% \*head[<for 'plain' page>]{<for 'scrheadings' page>}
\ohead[\lsstyle NUMBER~\liningfigures{\Number}]{}
\chead[\lsstyle\MakeUppercase{\Month}~\liningfigures{\Year}]{\small\scshape\MakeLowercase{\Journal}}
\ihead[\lsstyle VOLUME~\liningfigures{\Volume}]{}
%%% Fill in the blanks
\lehead{\small\thepage}
\rohead{\small\liningfigures{\thepage}}
\lohead{\small\Year]}
\rehead{\small[Vol.~\Volume}
\cohead{\small\scshape\MakeLowercase{\Title}}

% FOOTER and FOOTNOTE
\setlength{\footskip}{1.5\baselineskip}
\setkomafont{pagefoot}{\normalfont}
%%% \*foot[<for 'plain' page>]{<for 'scrheadings' page>}
\ofoot[\small\liningfigures{\thepage}]{}
\cfoot[]{}
\ifoot[]{}
\setfootnoterule[]{\textwidth}

% FONT
\usepackage[osf]{Baskervaldx}

% SECTION HEADINGS
\setcounter{secnumdepth}{1}
\let\raggedsection\centering
\renewcommand{\thesection}{\Roman{section}} 
\renewcommand\sectionformat{\liningfigures{\thesection}.\enskip}
\renewcommand{\thesubsection}{\thesection.\Roman{subsection}}
\renewcommand\subsectionformat{\thesubsection.\enskip}

\setkomafont{section}{\bfseries}
\RedeclareSectionCommand[
  %runin=false,
  afterindent=false,
  beforeskip=1\baselineskip,
  afterskip=.25\baselineskip]{section}

\setkomafont{subsection}{\normalfont\itshape}
\RedeclareSectionCommand[
  %runin=false,
  afterindent=false,
  beforeskip=1\baselineskip,
  afterskip=.25\baselineskip]{subsection}

\RedeclareSectionCommand[
  %runin=false,
  afterindent=false,
  beforeskip=.5\baselineskip,
  afterskip=.25\baselineskip]{subsubsection}
  
\RedeclareSectionCommand[
  runin=true,
  %afterindent=false,
  beforeskip=.5\baselineskip,
  afterskip=1em]{paragraph}

\RedeclareSectionCommand[
  runin=true,
  %afterindent=false,
  beforeskip=.5\baselineskip,
  afterskip=1em]{subparagraph}

% LETTRINE
\usepackage{lettrine}
\setlength{\emergencystretch}{2pt}
\def\drop #1#2 {% note the space before {
  \lettrine[lines=2,loversize=0.1,nindent=2pt]{#1}{#2} % a trailing space
}

\raggedbottom


% Computer Code Styleguide. 

\usepackage{listings}
\usepackage{xcolor}

% Define more subdued colors or remove colors altogether
\definecolor{commentcolor}{rgb}{0.5,0.5,0.5} % Gray for comments
\definecolor{backcolour}{rgb}{0.95,0.95,0.95} % Very light gray background, or consider removing the background color entirely

% Setup the listings package for a professional, academic style
\lstdefinestyle{hlrstyle}{
    backgroundcolor=\color{white},   % No background color for a clean look
    commentstyle=\color{commentcolor},
    keywordstyle=\bfseries,          % Bold for keywords, no color
    numberstyle=\tiny\color{black},  % Line numbers in black
    stringstyle=\ttfamily,           % Typewriter font for strings, no color
    basicstyle=\footnotesize\ttfamily, % Typewriter font
    breakatwhitespace=true,         
    breaklines=true,                 
    captionpos=b,                    
    keepspaces=true,                 
    numbers=left,                    
    numbersep=5pt,                  
    showspaces=false,                
    showstringspaces=false,
    showtabs=false,                  
    tabsize=2
}

\lstset{style=hlrstyle}
